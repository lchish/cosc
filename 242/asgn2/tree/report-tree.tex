\documentclass[12pt]{article} % Please don't change the font size!

\setlength{\textheight}{24cm}

\begin{document}

\pagestyle{empty} % we don't need a page number for a single page

\begin{center}
\section*{Alice's Adventures In Wonderland}
\end{center}

\section{Red Black Trees and Binary Search Trees}

Binary trees are a type of data structure that stores keys in a
tree structure where every node can have up to two children and
left nodes are smaller than the current node and right nodes are larger.
Binary search trees and red black trees both implement this data
structure in similar ways. Red black trees have the advantage that
after a new node is inserted or deleted from the tree it re-balances
itself so the depth of the tree stays nearly optimal. This helps
ensure that searching for a key is approximately O(log n), but
slows down the insertion and deletion time due to rebalancing.

Balancing a binary search tree is important if the type of data
to be inserted is not guaranteed to be completely random. If 
sequential data is inserted into a binary search tree it will
put all child nodes as left or right children, making the tree
height 0(n). Red black trees avoid this happening by rotating the
tree, allowing sequential data to be added without unbalancing the
tree.

For large random sets of data a binary search tree was slightly
faster than red black trees at creating the tree and nearly as
fast at searching.

\section{Improvements}

My tree implementation could have been improved by adding a delete
function allowing nodes in the tree to be removed without creating
a new tree. I could have also implemented other tree types such as
b-trees for camparison.

\subsection*{BE VERY CAREFUL NOT TO EXCEED 1 PAGE!}

\end{document}
